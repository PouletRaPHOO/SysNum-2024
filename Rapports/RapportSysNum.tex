\documentclass{article}

\title{Rapport architecture microprocesseur}
\author{HORNERO Baptiste \\ HERZLICH Raphaël \\ HELARY Axel}
\date{}

\usepackage[margin=.5in]{geometry}

\begin{document}

\maketitle

\part*{Netlist}

\subsection*{Compilation}

Afin de compiler le projet on utilisera : \verb|make| 

et pour compiler uniquement le scheduler on pourra utiliser : \verb|make schedule| 


\subsection*{Utilisation}

Afin de simuler la netlist sur le fichier \verb|/test/*.net| on va utiliser la commande : 
sh
\begin{verbatim}./netlist_simulator.byte test/*.net\end{verbatim}

Pour pouvoir utiliser des roms préinitialisée, on va pour chaque rom ident déclarée dans la netlist,
on va fournir dans le dossier test (l'emplacement des roms peut être changé dans le \verb|netlist_simulator.ml|)
un fichier \verb|ident.rom| selon la norme suivante:

Chaque ligne correspond à un \textbf{VBitArray}, ou 0 correspond à \verb|false| et 1 à \verb|true|, Autrement dit pour encoder la \textbf{ROM} r : 

\begin{tabular}{| c | c |}
    true & false \\
    false & false \\
    false & true \\
    false & true \\
    true & true \\
    false & false \\
    false & false \\
    false & false \\
\end{tabular}

on fournit le fichier \verb|r.rom| :

\begin{verbatim}
    10
    00
    01
    01
    11
    00
    00
    00
\end{verbatim}

De plus si aucun fichier n'est fourni, la ROM est initialisée entièrement remplie de false. De même si toutes les lignes ne sont pas spécifiées dans le fichier elles sont également initialisées à false. Ainsi pour initialiser la ROM précédente on peut également fournir le fichier \verb|r.rom| :

\begin{verbatim}
    10
    00
    01
    01
    11
\end{verbatim}

De même pour rentrer dans un input un \textbf{VBitArray} \verb|[|false;true;false;false|]| on rentrera \verb|Valeur de a (array de taille 4): 0100|

\subsection*{Spécifications }

\subsubsection*{Scheduler }

Le Scheduler commence par construire un graphe de dépendance entre toutes les variables. Puis ordonne les opérations selon un ordre topologique (s'il y en a un, sinon il renvoit une erreur \verb|Combinatorial_Cycle|).

Il y a cependant 2 choses notables à remarquer lors du passage du scheduler :

- On ne considère par l'opération \verb|REG| comme une opération de dépendance, puisque \verb|y = REG x| donne  à y la valeur de x à l'étape précédente (et donc n'est pas impactée par l'ordre de réassignation de x)
- Lors d'un appel \verb|y = RAM _ _ ra we wa data|, seule \verb|ra| est considérée comme une dépendance, puisque comme l'on va faire tout les write de \textbf{RAM} à la fin de chaque étape, toutes dépendance relative à l'écriture n'impacte pas l'ordre de simulation de l'opération.

\subsubsection*{Simulateur}
\textbf{Specifications particulières }

\underline{REG} :

- L'assignation d'un registre utilise l'environnement final de la passe précédente (du pre-process s'il s'agit de la première passe). Ainsi la lecture de \verb|y = REG x| à la passe *n* assignera à \textbf{y} la valeur de \textbf{x} à la fin de la passe \textbf{n-1}. 

\underline{NOT}:

- L'application de \textbf{NOT} sur un array est effectuée bit à bit.

\underline{BINOP}:

- Une binop entre deux \textbf{VBit} est faite de façon normale

- Une binop entre un \textbf{VBit b} et un \textbf{VBitArray a}  est faite bit à bit sur les éléments de \textbf{a} et \textbf{b}

- Une binop entre deux \textbf{VBitArray a1,a2} renvoie un tableau de la taille minimum entre les deux arrays rempli des opérations bit à bit. 

\underline{CONCAT}:

- Une concat entre deux \textbf{VBitArray} agit de façon intuitive

- Le reste des opérations possibles considèrent les \textbf{VBit} commes des \textbf{VBitArray} de taille 1

\underline{SELECT}:

- Une select sur un \textbf{VBit} n'est légale seulement si on sélectionne l'élément 0, dans ce cas elle renvoie le \textbf{VBit}

- Une select sur des \textbf{VBitArray} agit de façon intuitive

\underline{SLICE}:

- Un slice n'est légal que sur un \textbf{VBitArray}

\underline{MUX}:

- On va considérer que l'argument de choix pointe vers le premier élément dans deux cas :

  - C'est un \textbf{VBit true}

  - C'est un \textbf{VBitArray} entièrement remplit de true

- Dans tout les autres cas on renverra le deuxième élement

\underline{ROM}:

- Une addresse est forcément un \textbf{VBitArray} de taille inférieure ou égale à \verb|address_size|

\underline{RAM}:

- Pareil que pour \verb|ROM|


\part*{Processeur}

\section*{Architecture du processeur}

L'architecture du processeur est une architecture proche de l'architecture x86,
les instructions sont codées sur 32 bits. On a 16 registres accessibles,
numérotés de 0 à 15, chacun de taille 32. On ajoute un register de 32
bits correspondant à la position du pointeur sur la ROM, qu'on appelle P.
On se munit de trois flags : ZF (dernière opération a renvoyé 0 ou non), SF
(positivité de la dernière opération), OF (overflow de la dernière
opération arithmétique) tous de taille 1.


\section*{Instructions}
 
Le set d'instructions est le suivant:

\begin{center}

\begin{tabular}{| c || c || c | c |}
    Instruction & Encodage & Description & Arguments\\ \hline
    NOP & 0000 0000 & No opération &\\ \hline
    ADD & 0001 0001 & Addition & rs1 rs2\\ \hline
    SUB & 0001 0010 & Soustraction & rs1 rs2 \\ \hline
    MUL & 0001 0011 & Multiplication & rs1 rs2 \\ \hline
    AND & 0010 0001 & Et logique & rs1 rs2\\ \hline
    OR  & 0010 0010 & Ou logique & rs1 rs2\\ \hline
    XOR & 0010 0011 & Xor logique & rs1 rs2\\ \hline
    NOT & 0011 0001 & Non & rs1\\ \hline
    SLL & 0011 0010 & Décalage gauche logique & rs1 \\ \hline
    SRL & 0011 0011 & Décalage droite logique & rs1\\ \hline
\end{tabular}

et :

\begin{tabular}{| c || c | c|| c |  c  |}
    Instruction & Encodage & Description & Arguments & Description
formelle \\ \hline
    MOV   & 0110 0001 & Met un registre à la valeur d'un autre registre &
rs1 rs2  & rs1 $\gets$ rs2\\ \hline
    MOVI  & 0111 0001 & Met une valeur immédiate dans un registre & rs1
immediate & rs1 $\gets$ immediate\\\hline
    CMP   & 1000 0001 & Compare deux valeurs & rs1 rs2& Mise à jour des
flags\\ \hline
    JMP   & 0100 0001 & Jump à une valeur & immediate & P $\gets$ immediate\\ \hline
    JNE   & 0100 0010 & Jump si non égal (ZF = 0) & immediate & Si ZF =0
alors P$\gets$ immediate\\ \hline
    JE    & 0100 0011 & Jump si égal (ZF = 1) & immediate&  Si ZF = 1
    alors P $\gets$ immediate \\ \hline
    JGE   & 0100 0100 & Jump si plus grand ou égal & immediate& Si OF=SF
alors P $\gets$ immediate\\ \hline
    LOAD  & 0101 0001 & Lit dans la ram & rs1 rs2& rs1 $\gets$ R[rs2]\\ \hline
    STORE & 0101 0010 & Stocke dans la ram & rs1 rs2& R[rs2] $\gets$ rs1\\ \hline
    LOADFIX  & 0101 0011 & Lit dans la ram & rs1 immediate& rs1 $\gets$ R[immediate]\\ \hline
    STOREFIX & 0101 0100 & Stocke dans la ram & rs1 immediate& R[immediate] $\gets$ rs1\\ \hline
\end{tabular}


\end{center}

\end{document}
