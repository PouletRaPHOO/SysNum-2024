\documentclass{beamer}

\title{Présentation Systèmes Numériques}
\date{14 janvier 2025}
\author{HERZLICH Raphaël, HELARY Axel, HORNERO Baptiste}

\begin{document}

    \frame{\titlepage}

    \begin{frame}{Sommaire}
        \tableofcontents[hideallsubsections]
    \end{frame}


\section{Principe général}
    \begin{frame}
        \tableofcontents[currentsection, hideothersubsections]
    \end{frame}

    \subsection{Architecture}
    \begin{frame}{Architecture}
        On implémente:
        \begin{itemize}
            \item 16 registres de taille 32
            \item P qui donne la position du pointeur sur la ROM
            \item ZF, SF, OF trois flags de taille 1
        \end{itemize}
    \end{frame}

    \subsection{Jeu d'instruction}
    \begin{frame}{Jeu d'instruction}

        \begin{tabular}{| c || c || c | c |}
            Instruction & Encodage & Description & Arguments\\ \hline
            NOP & 0000 0000 & No opération &\\ \hline
            ADD & 0000 0001 & Addition & rs1 rs2\\ \hline
            SUB & 0000 0010 & Soustraction & rs1 rs2 \\ \hline
            MUL & 0000 0011 & Multiplication & rs1 rs2 \\ \hline
            AND & 0000 0100 & Et logique & rs1 rs2\\ \hline
            NOT & 0000 0101 & Non & rs1\\ \hline
            OR  & 0000 0110 & Ou logique & rs1 rs2\\ \hline
            XOR & 0000 0111 & Xor logique & rs1 rs2\\ \hline
            SLL & 0000 1000 & Décalage gauche logique & rs1 \\ \hline
            SRL & 0000 1001 & Décalage droite logique & rs1\\ \hline
        \end{tabular}
    \end{frame}

    \begin{frame}{Jeu d'instruction}

        \begin{center}
        \begin{tabular}{| c || c | c |}
            Instruction & Encodage & Description formelle \\ \hline
            MOV   & 0000 1010 & rs1 $\gets$ rs2\\ \hline
            MOVI  & 0000 1011 & rs1 $\gets$ immediate\\\hline
            JMP   & 0000 1100 & P $\gets$ immediate\\ \hline
            CMP   & 0000 1101 & \\ \hline
            JE    & 0000 1110 &  Si ZF = 1 alors P $\gets$ immediate \\ \hline
            JNE   & 0000 1111 & Si ZF =0 alors P$\gets$ immediate\\ \hline
            JGE   & 0001 0000 & Si OF=SF alors P $\gets$ immediate\\ \hline
            LOAD  & 0001 0001 & rs1 $\gets$ R[rs2]\\ \hline
            STORE & 0001 0010 & R[rs2] $\gets$ rs1\\ \hline
            LOADFIX  & 0001 0001 & rs1 $\gets$ R[immediate]\\ \hline
            STOREFIX & 0001 0010 & R[immediate] $\gets$ rs1\\ \hline
        \end{tabular}
    \end{center}

    \end{frame}

\section{Assembleur}
    \begin{frame}
        \tableofcontents[currentsection, hideothersubsections]
    \end{frame}
\section{Horloge}
    \begin{frame}
        \tableofcontents[currentsection, hideothersubsections]
    \end{frame}

\subsection{Utilité}

\begin{frame}{Ce que l'on fait}
    Notre horloge permet de:
    \begin{itemize}
        \item passer les secondes, les minutes, les heures, les jours, les jours de la semaine, les mois et les années
        \item détecter les années bissextiles et de modifier le nombre de jours de février en conséquence
        \item choisir la date de début
    \end{itemize}

    Notre horloge débutera à l'année 1970.
\end{frame}

\begin{frame}{Les problèmes}
    Nous avons rencontré les problèmes suivants:
    \begin{itemize}
        \item Netlist trop grande en utilisant des boucles for
        \item Netlist trop grande à cause des multiplications
        \item Implémentation peu efficace de SLL 
    \end{itemize}
\end{frame}

\subsection{Spécificité}

\end{document}